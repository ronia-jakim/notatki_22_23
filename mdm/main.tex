\documentclass{article}

\usepackage{../uni-notes}

\title {Kombinatoryka $\&$ teoria grafów}
\author{a fish}
\date {21.03.2137}

\begin{document}

\maketitle

\newpage

\section*{SYLABUS - MDM:}
\indent A. ELEMENTY ALGEBRY I TEORII LICZB\medskip\\
\indent\indent - Funkcje całkowitoliczbowe, arytmetyka modularna, operacje sufit i podłoga zaokrąglania liczb rzeczywistych, algorytm mergesort\\
\indent\indent - Asymptotyka funkcji liczbowych z uwzględnieniem zastosować w szacowaniu złożoności czasowej algorytów\\
\indent\indent - Podzielność liczb, algorytm Euklidiesa\\
\indent\indent - Liczby Fibonacciego\\
\indent\indent - Liczby pieriwsze i względnie pierwsze. Rozkład na czynniki. Funckja Eulera. Chińskie twierdzenie o resztach. Twierdzenie Eulera.\bigskip\\
\indent B. KOMBINATORYKA\medskip\\
\indent\indent - Rozmieszczenia, permutacje, kombinacje, podziały (zbioru, liczbyb), Lemat Burnside'a\\
\indent\indent - Metody generowania prostych obiektów kombinatorycznych\\
\indent\indent - Przykłady prostych problemów definiowanych rekurencyjnie\\
\indent\indent - Rozwązywanie równań rekurencyjnych, funkcje tworzące\\
\indent\indent - Liczby Catalana\\
\indent\indent - Zasada włączania i wyłączania\bigskip\\
\indent C. TEORIA GRAFÓW\medskip\\
\indent\indent - Definicja i przykłady grafów, grafy pełne, dwudzielne skierorwane, stopień wierzchołka\\
\indent\indent - Drogi i cykle w grafach: grafy spójne i dwudzielne\\
\indent\indent - Drzewa - równoważność różnych definicjii\\
\indent\indent - Komputeroaw reprezentacja grafów\\
\indent\indent - Metody BFS i DFS przeszukiwania grafów\\
\indent\indent - Minimalne drzewa rozpinające - algorytmy Kruskala i Prima-Dijkstry\\
\indent\indent - Przechodznie domknięcie: algorytmy Dijkstry i Warshalla. Złożoność problemu\\
\indent\indent - Cykle i drogi Eulera\\
\indent\indent - Cykle i drogi Hamiltowa, twierdzenie Ore i wielomianowa redukcja problemu drogi do cyklu i odwrotnie\\
\indent\indent - Przepływy w sieciach\\
\indent\indent - Kolorowanie grafów: zastosowanie - planowanie sesji egzaminacyjnej. Algorytm sekwencyjny i twierdzenie o 5-kolorowaniu grafów planarnych.

\newpage

\section*{SYLABUS - teoria grafów:}
1. Basic concepts: graphs, paths and cycles, complete andbipartite graphs\\
2. Matchings: Hall's Marriage theorem and its variations\\
3. Forbidden subgraphs: complete bipartite and r-partite subgraphs, chromatic numbers, Tur"an's thorem, asymptotic behaviour og edge density, Erd"os-Stone theorem\\
4. Hamiltonian cycles (Dirac's Theorem), Eulerian circuits\\
5. Connectivity: connected and k-connected graphs, Menger's theorem\\
6. Ramsey theory: edge colourings of graphs, Ramsey's theorem and its variations, asymptotic bounds on Ramsey numbers\\
7. Planar graphs and colourings: statements of Kuratowski's and Four Colour theorems, proof of Five Colour theorem, graphs on other surfaces and Euler chracteristics, chromatic polynomial, edge colourings and Vizing's theorem\\
8. Random graphs: further asymptotic bounds on Ramsey numbers, Zarankiewicz numbers and their bounds, graphs of large firth and high chromatic number, cmplete subgraphs in random graphs.\\
9. Algebraic methods: adjavenvy matrix and its eigenvalues, strongly regular graphs, Moore graphs and their existence.

\newpage


\tableofcontents

\newpage


\end{document}