\subsection{Hierarchia - asymptotyka}
$$f(n)\prec g(n)\iff\lim\limits_{n\to\infty}{f(n)\over g(n)}=0$$
Czyli $f(n)$ rośnie wolniej niż $g(n)$. Na przykład $n\prec n^2$. Podstawowa hierarchia to dla $0<\varepsilon<1\leq c$:
$$1\prec \log\log n\prec\log n\prec n^\varepsilon\prec n^c\prec n^{\log n}\prec c^n\prec n^n\prec c^{c^n}.$$
Wszystkie te funkcje przy $n\to\infty$ dążą do nieskończoności, kluczowe więc nie jest określenie czy to robią, a raczej {\color{acc}jak szybko do $\infty$ dążą}. Alternatywnie, możemy porównywać odwrotności funkcji i jak szybko one dążą do zera, nigdy go nie osiągając.\bigskip

$$e^{f(n)}\prec e^{g(n)}\iff \lim\limits_{n\to \infty}(f(n)-g(n))=-\infty$$

Dwie funkcje są do siebie {\color{def}asymptotyczne}, jeżeli mają {\color{acc}ten sam współczynnik}. Piszemy wtedy $f(n)\asymp g(n)$, czyli
$$f(n)\asymp g(n)\iff |f(n)|\leq C\cdot |g(n)|\;\land\;|g(n)|\leq C\cdot|f(n)|,$$
dla pewnej stałej $C$ oraz dostatecznie dużych $n$. Zachodzi to na przykład dla wielomianów tego samego stopnia.\bigskip

{\color{def}Klasa funkcji logarytmiczno-wykładniczych} jest zdefiniowana rekurencyjnie jako najmniejsza rodzina $L$ (to powinno być gotyckie $L$, ale coś nie działą) spełniająca\smallskip\\
\indent 1. dla każdego $\alpha\in \R$ funkcja $f(n)=\alpha$ należy do $L$\\
\indent 2. funkcja tożsamościowa $f(n)=n$ należy do $L$\\
\indent 3. jeżeli $f(n)$ i $g(n)$ należą do $L$, to również $f(n)-g(n)$ należy do $L$\\
\indent 4. jeżeli $f(n)$ należy do $L$ oraz jest dodatnia od pewnego momentu, to również $\ln(f(n))$ należy do $L$.\medskip

{\color{def}Główne twierdzenie Hardy'ego} mówi, że jeżeli $f(n), g(n)\in L$ to zachodzi jedna z możliwości:\smallskip\\
\indent 1. $f(n)\prec g(n)$\\
\indent 2. $g(n)\prec f(n)$\\
\indent 3. $f(n)\asymp g(n)$.

%=============================================================================%

\subsection{Big $O$ notation}
Zapis $\color{def}f(n)=O(g(n))$ oznacza, że $\color{def}|f(n)|\leq C|g(n)|$ dla wszystkich $n$. Czyli oznacza, że $O(5)$ jest liczbą której wartość bezwzględa po pomnożeniu przez jakąś stałą jest nie większa niż $5$.\bigskip

Bardzo często narzucamy na notację $O$ pewne ograniczenia, na przykład powiedzenie, że
$$f(n)=O(g(n)),\quad n\to \infty$$
oznacza, że warunek jest spełniony dla $n$ bardzo bliskich $\infty$, a o inne $n$ nie dbamy. Czyli tak naprawdę nakładamy dwie stałe: $C$ do mnożenia i $n_0$ od którego warunek zaczyna być spełniany.\bigskip

Uwaga, znak $=$ w kontekscie notacji dużego $O$ to lekkie nadużycie. Notacja $f(n)=O(g(n))$ oznacza tylko, że $f(n)$ należy do pewnego zbioru funkcji takich, że istnieje $C$ takie, że $f(n)\leq C|g(n)|$. Dlatego też piszemy $O$ po prawej stronie równania nie po lewej - znak $=$ to tak naprawdę leniwe $\subseteq$.

\subsection{Notacja Duże $\Omega$}
Używana jest do dolnych ograniczeń funkcj, tzn
$$f(n)=\Omega(g(n))\iff |f(n)|\geq C|g(n)|$$
dla pewnego $C>0$. Mamy więc 
$$f(n)=\Omega(g(n))\iff g(n)=O(f(n)).$$
Czyli na przykład algorytm sortujący w czasie $\Omega(n^2)$ jest o wiele bardziej nieefektywny niż $O(n\log n)$.

\subsection{Notacja Dużego $\Theta$}
Określa dokładny porządek przyrostu:
$$f(n)=\Theta(g(n))\iff \begin{matrix}f(n)=O(g(n))\\f(n)=\Omega(n)\end{matrix}.$$
Czyli $f(n)=\Theta(g(n))\iff f(n)\asymp g(n)$.

\subsection{Notacja małego $o$}
Odpowiada relacji $f(n)\prec g(n)$. Dodatkowo, mamy
$$f(n)\sim g(n)\iff f(n)=g(n)+o(g(n)).$$

\subsection{Reguły notacji dużego $O$}
Ciąg bardzo fajnych i przyjemnych wzorków które trzeba się naumieć c:
$$n^a=O(n^b)\quad a\leq b$$
$$O(f(n))+O(g(n))=O(|f(n)|+|g(n)|)$$
$$f(n)=O(f(n))$$
$$c\cdot O(f(n))=O(f(n))$$
$$O(O(f(n)))=O(f(n))$$
$$O(f(n))O(g(n))=O(f(n)g(n))$$
$$O(f(n)g(n))=f(n)O(g(n))$$

Jeżeli suma
$$S(z)=\sum\limits_{n\geq0}a_nz^n$$
jest bezwzględnie zbieżna dla pewnego $z_0\in \Z$, to
$$S(z)=O(1)\quad (\forall\;z)\;|z|\leq|z_0|,$$
bo
$$S(z)=\sum\limits_{n\geq 0}a_nz^n\leq\sum\limits_{n\geq0}a_nz_0^n=C<\infty$$

Przybliżenie asymptotyczne ma {\color{def}błąd bezwzględy} równy $O(g(n))$ jeżeli jest ono postaci $f(n)+O(g(n))$, gdzie $f(n)$ nie zawiera $O$. {\color{def}Błąd względny} jest równy $O(g(n))$ gdy jest ono z kolei postaci $f(n)(1+O(g(n)))$.