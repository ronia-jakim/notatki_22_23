\documentclass{article}[13pt]

\usepackage{../uni-notes}
\usepackage{multicol}

\title {Kombinatoryka $\&$ teoria grafów}
\author{by a fish}
\date {21.03.2137}

\begin{document}

\maketitle

\newpage

\section*{SYLABUS - MDM:}
\indent A. ELEMENTY ALGEBRY I TEORII LICZB\medskip\\
\indent\indent - Funkcje całkowitoliczbowe, arytmetyka modularna, operacje sufit i podłoga zaokrąglania liczb rzeczywistych, algorytm mergesort\\
\indent\indent - Asymptotyka funkcji liczbowych z uwzględnieniem zastosować w szacowaniu złożoności czasowej algorytów\\
\indent\indent - Podzielność liczb, algorytm Euklidiesa\\
\indent\indent - Liczby Fibonacciego\\
\indent\indent - Liczby pieriwsze i względnie pierwsze. Rozkład na czynniki. Funckja Eulera. Chińskie twierdzenie o resztach. Twierdzenie Eulera.\bigskip\\
\indent B. KOMBINATORYKA\medskip\\
\indent\indent - Rozmieszczenia, permutacje, kombinacje, podziały (zbioru, liczbyb), Lemat Burnside'a\\
\indent\indent - Metody generowania prostych obiektów kombinatorycznych\\
\indent\indent - Przykłady prostych problemów definiowanych rekurencyjnie\\
\indent\indent - Rozwązywanie równań rekurencyjnych, funkcje tworzące\\
\indent\indent - Liczby Catalana\\
\indent\indent - Zasada włączania i wyłączania\bigskip\\
\indent C. TEORIA GRAFÓW\medskip\\
\indent\indent - Definicja i przykłady grafów, grafy pełne, dwudzielne skierorwane, stopień wierzchołka\\
\indent\indent - Drogi i cykle w grafach: grafy spójne i dwudzielne\\
\indent\indent - Drzewa - równoważność różnych definicjii\\
\indent\indent - Komputeroaw reprezentacja grafów\\
\indent\indent - Metody BFS i DFS przeszukiwania grafów\\
\indent\indent - Minimalne drzewa rozpinające - algorytmy Kruskala i Prima-Dijkstry\\
\indent\indent - Przechodznie domknięcie: algorytmy Dijkstry i Warshalla. Złożoność problemu\\
\indent\indent - Cykle i drogi Eulera\\
\indent\indent - Cykle i drogi Hamiltowa, twierdzenie Ore i wielomianowa redukcja problemu drogi do cyklu i odwrotnie\\
\indent\indent - Przepływy w sieciach\\
\indent\indent - Kolorowanie grafów: zastosowanie - planowanie sesji egzaminacyjnej. Algorytm sekwencyjny i twierdzenie o 5-kolorowaniu grafów planarnych.

\newpage

\section*{SYLABUS - teoria grafów:}
1. Basic concepts: graphs, paths and cycles, complete andbipartite graphs\\
2. Matchings: Hall's Marriage theorem and its variations\\
3. Forbidden subgraphs: complete bipartite and r-partite subgraphs, chromatic numbers, Tur"an's thorem, asymptotic behaviour og edge density, Erd"os-Stone theorem\\
4. Hamiltonian cycles (Dirac's Theorem), Eulerian circuits\\
5. Connectivity: connected and k-connected graphs, Menger's theorem\\
6. Ramsey theory: edge colourings of graphs, Ramsey's theorem and its variations, asymptotic bounds on Ramsey numbers\\
7. Planar graphs and colourings: statements of Kuratowski's and Four Colour theorems, proof of Five Colour theorem, graphs on other surfaces and Euler chracteristics, chromatic polynomial, edge colourings and Vizing's theorem\\
8. Random graphs: further asymptotic bounds on Ramsey numbers, Zarankiewicz numbers and their bounds, graphs of large firth and high chromatic number, cmplete subgraphs in random graphs.\\
9. Algebraic methods: adjavenvy matrix and its eigenvalues, strongly regular graphs, Moore graphs and their existence.

\newpage


\tableofcontents

\newpage

\section{Elementy algebryi teorii liczb}
\subsection{Podłoga i sufit}

Reguły funkcji podłoga i sufit:\smallskip\\
\indent 1. $\lfloor x\rfloor =n\iff n\leq x< n+1$\\
\indent 2. $\lfloor x\rfloor = n\iff x-1<n\leq x$\\
\indent 2. $\lceil x\rceil = n\iff n-1<x\leq n$\\
\indent 3. $\lceil x\rceil = n\iff x\leq n< x+1$\bigskip

Więcej hardocorowych własności:
$$\lfloor x+n\rfloor = \lfloor x\rfloor+n$$
ale dla mnożenia to już nie zadziała.\medskip

Dla liczb całkowitych zachodzi:
$$x<n\iff \lfloor x\rfloor < n$$
$$n<x\iff n < \lceil x\rceil$$
$$x\leq n\iff \lceil x\rceil$$
$$n\leq x\iff n\leq \lfloor x\rfloor$$

{\color{def}Część ułamkowa} to różnica $x-\lfloor x\rfloor$. Oznaczamy $\{x\}$, chyba że gdzieś obok pojawiają się singletony. Wtedy nie oznaczamy. Simple.

\podz{sep}

$$\lfloor \sqrt{\lfloor x\rfloor}\rfloor=\lfloor\sqrt{x}\rfloor$$

W pierwszej kolejności chcemy pozbyć się zewnętrznych nawiasów i pierwiastka kwadratowego, potem usunąć nawiasy wewnętrzne. Na koniec dodajemy z powrotem pierwiatek i nawiasy zewnętrzne:
$$\lfloor \sqrt{\lfloor x\rfloor}\rfloor\to \sqrt{\lfloor x\rfloor}\to\lfloor x\rfloor\to x\to \sqrt{x}\to \lfloor \sqrt{x}\rfloor$$

\medskip

Niech $m=\lfloor \sqrt{\lfloor x\rfloor}\rfloor$. Z wcześniej ustalonych reguł wynika, że 
$$m\leq \sqrt{\lfloor x\rfloor} < m+1.$$
Ponieważ wszystkie trzy wyrażenia są nieujemne, możemy podnieść je do kwadratu:
$$m^2\leq \lfloor x\rfloor < (m+1)^2.$$
Ponieważ zarówno $m$ jak i $m+1$ są liczbami całkowitymi, to możemy pozbyć się nawiasów kwadratowych, otrzymując
$$m^2\leq x < (m+1)^2.$$
Obie strony nadal są nieujemne, możemy je więc spierwiastkować, żeby otrzymać
$$m\leq \sqrt{x}<m+1$$
a więc
$$m=\lfloor \sqrt{x}\rfloor$$
i to jest to, co chceliśmy otrzymać.
\kdowod

Możemy tę równość uogólnić dla dowolnej funkcji $f$ takiej, że $f(x)\in \Z\implies x\in \Z$:
$$\lfloor f(x)\rfloor=\lfloor f(\lfloor x\rfloor)\rfloor\;\land\; \lceil f(x)\rceil=\lceil f(\lceil x\rceil)\rceil.$$

Szczególnym przypadkiem tego twierdzenia jest
$$\Big\lfloor {x+m\over n}\Big\rfloor=\Big\lfloor {\lfloor x\rfloor+m\over n}\Big\rfloor\;\land\;\Big\lceil {x+m\over n}\Big\rceil=\Big\lceil {\lceil x\rceil+m\over n}\Big\rceil$$

\bigskip

\podz{sep}
\bigskip

{\color{def}Widmo liczby rzeczywistej} $\alpha$ to nieskończony zbiór liczb całkowitych z powtórzeniami:
$$Spec(\alpha) = \{\lfloor\alpha\rfloor, \lfloor2\alpha\rfloor, \lfloor3\alpha\rfloor, ...\}.$$

Ilość elementów $Spec(\alpha)$ nie większych niż $n$ wynosi:
\begin{align*}
    N(\alpha, n)=\sum\limits_{k>0}(\lfloor k\alpha\leq n\rfloor)=\sum\limits_{k>0}(\lfloor k\alpha<n+1\rfloor)=\sum\limits_{k>0}(k\alpha<n+1)=\lceil {(n+1)\over\alpha}-1\rceil
\end{align*}

\newpage

\section{Basic concepts of graph theory}

\subsection{Graphs}

{\color{def}Graph} ($G=(V, E)$) - a structure made up of {\color{acc}vertices} $(V)$ that are connected in pairs with {\color{acc}edges} $(E)$.\medskip

{\color{def}Multigraph} - a graph where two vertices are allowed to have more than one egde connecting them.\medskip

If a vertex is allowed to be connected to itself, then the graph is called a {\color{def}graph with loops} and the edge that connects the vertex to itself is known as a {\color{def}loop}.\medskip

{\color{def}Adjacency relation} - is the symmetric relation of pairings between vertices of an undirected graph. It is used to construct an {\color{acc}adjacency matrix} that is another form of representing graphs.
\begin{multicols}{2}
    {\color{back}hjhg}
\pgraf
    \node (p1) at (0, 0) {1 $\bullet$};
    \node (p2) at (1, -0.5) {$\bullet$ 2};
    \node (p3) at (1, 0.5) {$\bullet$ 3};
    \draw[sep, very thick] (p1)--(p2);
    \draw[sep, very thick] (p1)--(p3);
\kgraf

\columnbreak
{\color{back}jhv}
\begin{align*}
    &\begin{matrix}\quad1 && 2 && 3\end{matrix}\\
    &\begin{pmatrix}
        0 && 1 && 1\\
        1 && 0 && 0\\
        1 && 0 && 0
    \end{pmatrix}
\end{align*}

\end{multicols}\bigskip

{\color{def}Directed graph} is a graph in which edges have orientation. Here, the set of edges contains ordered pairs of vertices. However, this definition does not allow multiple edges between two vertices. To fix this problem, we introduce another object, $\phi$, that is a mapphing of edges to ordered pairs of vartices. To avoid confusion, we call such graph a {\color{acc}directed multigraph} $(G=(V, E, \phi))$.\medskip

{\color{def}Mixed graph} is a graph that allows both directed and undirected edges.\smallskip

{\color{def}Weigthed graph} is a graph in which each edge has a value assigned to it.\smallskip

{\color{def}Oriented graph} is a directed graph where each edge has a set orientation, that is if an edge $\parl x, y\parr$ exists, there cannot be an edge $\parl y, x\parr$\smallskip

{\color{def}Regular graph} is a graph in which each vertex has the same number of neighbours ({\color{acc}degree}).\smallskip

{\color{def}Complete graph} is a graph where every pair of vertices is connected with an edge.\bigskip

\begin{multicols}{2}
    {\color{def}Tree} is a graph in which any two vertices are connected by exactly one path.\smallskip
    
    {\color{def}Polytree} is a graph whose underlying graph is a tree. For example on the right is a polytree in which subgraph (A, D, F, G) is a tree.
    \pgraf
        \node (a) at (0, 0) {A};
        \node (b) at (1, 0) {B};
        \node (c) at (-0.5, -1) {C};
        \node (d) at (0.5, -1) {D};
        \node (e) at (1.5, -1) {E};
        \node (f) at (0, -2) {F};
        \node (g) at (1, -2) {G};
        \draw [thick, ->] (a)--(c);
        \draw[thick, ->] (a)--(d);
        \draw[thick, ->] (b)--(d);
        \draw[thick, ->] (b)--(e);
        \draw[thick, ->](d)--(f);
        \draw[thick, ->](d)--(g);
    \kgraf
\end{multicols}

\subsection{Paths}

A pair of vertices $x, y$ is {\color{def}connected} if there can be found a collection of edges so that they are have connected ends and going through them leads from $x$ to $y$ and vice versa. Such a collection is called a {\color{def}path}.

A graph is {\color{def}connected} if each two vertices are connected. A stronger condition, each two vertices are connected with directed edges, makes a graph {\color{def}strongly connected}.\medskip

\podz{sep}\medskip

{\color{def}Chromatic number} - the smallest number of colors needed to color a graph so that every two vertices of an edge have different colors. For example the following graph has chromatic number 3:
\pgraf
    \node (p1) at (0, 0) {$\color{red}\bullet$};
    \node (p2) at (1, 0) {$\color{blue}\bullet$};
    \node (p3) at (-1, 0.5) {$\color{green}\bullet$};
    \node (p4) at (-1, -0.5) {$\color{blue}\bullet$};
    \draw[thick] (p1)--(p2);
    \draw[thick] (p1)--(p3);
    \draw[thick] (p1)--(p4);
    \draw[thick] (p3)--(p4);
\kgraf

\podz{sep}\medskip

{\color{def}Bipartite graph} is a simple graph where vertex set can be partioned into two sets. Alternatively, it is a graph with chromatic number 2.\medskip

{\color{def}Planar graph} is a graph that can be drawn on a plane so that no two edges intersect.

\subsection{Cycles}

{\color{def}Cycle} - non-empty trail in which only the first and last vertices are equal. If more than just the first and last vertices repeat in a cycle, then it is called a {\color{def}circuit}.\smallskip

{\color{def}Cycle graph} of order $n$ is a graph where $n$ vertices create a cycle. They are connected graphs with vertices of degree 2. If no cycles exist in a graph, then it is called an {\color{def}acyclic graph}.\bigskip

{\color{def}Chordless cycle} is a cycle in which no two vertices are connected by an edge that itself does not belong in the cycle, for example $(A, B, C, D, E, F)$ form an chordless cycle
\pgraf
    \node (a) at (0, 0) {A};
    \node (b) at (1, 0.5) {B};
    \node (c) at (1.5, -0.5) {C};
    \node (d) at (1, -1.5) {D};
    \node (e) at (0, -2.1) {E};
    \node (f) at (-0.5, -1) {F};
    \node (g) at (2.5, -0.5) {G};
    \node (h) at (2.5, -1.5) {H};
    \draw[title-color, thick] (a)--(b);
    \draw[title-color, thick] (c)--(b);
    \draw[title-color, thick] (c)--(d);
    \draw[title-color, thick] (d)--(e);
    \draw[title-color, thick] (e)--(f);
    \draw[title-color, thick] (a)--(f);
    \draw[thick] (c) -- (g);
    \draw[thick] (c)--(h);
\kgraf
Whereas in the next example, $(A, B, C, D, E, F)$ do not form a chordless cycle - edge $\{A, D\}$ is a {\color{acc}chord}.
\pgraf
    \node (a) at (0, 0) {A};
    \node (b) at (1, 0.5) {B};
    \node (c) at (1.5, -0.5) {C};
    \node (d) at (1, -1.5) {D};
    \node (e) at (0, -2.1) {E};
    \node (f) at (-0.5, -1) {F};
    \node (g) at (2.5, -0.5) {G};
    \node (h) at (2.5, -1.5) {H};
    \draw[title-color, thick] (a)--(b);
    \draw[title-color, thick] (c)--(b);
    \draw[title-color, thick] (c)--(d);
    \draw[title-color, thick] (d)--(e);
    \draw[title-color, thick] (e)--(f);
    \draw[title-color, thick] (a)--(f);
    \draw[thick] (a) -- (d);
    \draw[thick] (c) -- (g);
    \draw[thick] (c)--(h);
\kgraf

{\color{def}Girth} of a graph is the length of its shortest chordless cycle. {\color{def}Cages} are regular graphs with as few vertices as possible for its girth.


\end{document}