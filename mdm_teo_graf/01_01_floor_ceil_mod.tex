\section{Elementy algebryi teorii liczb}
\subsection{Podłoga i sufit}

Reguły funkcji podłoga i sufit:\smallskip\\
\indent 1. $\lfloor x\rfloor =n\iff n\leq x< n+1$\\
\indent 2. $\lfloor x\rfloor = n\iff x-1<n\leq x$\\
\indent 2. $\lceil x\rceil = n\iff n-1<x\leq n$\\
\indent 3. $\lceil x\rceil = n\iff x\leq n< x+1$\bigskip

Więcej hardocorowych własności:
$$\lfloor x+n\rfloor = \lfloor x\rfloor+n$$
ale dla mnożenia to już nie zadziała.\medskip

Dla liczb całkowitych zachodzi:
$$x<n\iff \lfloor x\rfloor < n$$
$$n<x\iff n < \lceil x\rceil$$
$$x\leq n\iff \lceil x\rceil$$
$$n\leq x\iff n\leq \lfloor x\rfloor$$

{\color{def}Część ułamkowa} to różnica $x-\lfloor x\rfloor$. Oznaczamy $\{x\}$, chyba że gdzieś obok pojawiają się singletony. Wtedy nie oznaczamy. Simple.

\podz{sep}

$$\lfloor \sqrt{\lfloor x\rfloor}\rfloor=\lfloor\sqrt{x}\rfloor$$

W pierwszej kolejności chcemy pozbyć się zewnętrznych nawiasów i pierwiastka kwadratowego, potem usunąć nawiasy wewnętrzne. Na koniec dodajemy z powrotem pierwiatek i nawiasy zewnętrzne:
$$\lfloor \sqrt{\lfloor x\rfloor}\rfloor\to \sqrt{\lfloor x\rfloor}\to\lfloor x\rfloor\to x\to \sqrt{x}\to \lfloor \sqrt{x}\rfloor$$

\medskip

Niech $m=\lfloor \sqrt{\lfloor x\rfloor}\rfloor$. Z wcześniej ustalonych reguł wynika, że 
$$m\leq \sqrt{\lfloor x\rfloor} < m+1.$$
Ponieważ wszystkie trzy wyrażenia są nieujemne, możemy podnieść je do kwadratu:
$$m^2\leq \lfloor x\rfloor < (m+1)^2.$$
Ponieważ zarówno $m$ jak i $m+1$ są liczbami całkowitymi, to możemy pozbyć się nawiasów kwadratowych, otrzymując
$$m^2\leq x < (m+1)^2.$$
Obie strony nadal są nieujemne, możemy je więc spierwiastkować, żeby otrzymać
$$m\leq \sqrt{x}<m+1$$
a więc
$$m=\lfloor \sqrt{x}\rfloor$$
i to jest to, co chceliśmy otrzymać.
\kdowod

Możemy tę równość uogólnić dla dowolnej funkcji $f$ takiej, że $f(x)\in \Z\implies x\in \Z$:
$$\lfloor f(x)\rfloor=\lfloor f(\lfloor x\rfloor)\rfloor\;\land\; \lceil f(x)\rceil=\lceil f(\lceil x\rceil)\rceil.$$

Szczególnym przypadkiem tego twierdzenia jest
$$\Big\lfloor {x+m\over n}\Big\rfloor=\Big\lfloor {\lfloor x\rfloor+m\over n}\Big\rfloor\;\land\;\Big\lceil {x+m\over n}\Big\rceil=\Big\lceil {\lceil x\rceil+m\over n}\Big\rceil$$

\bigskip

\podz{sep}
\bigskip

{\color{def}Widmo liczby rzeczywistej} $\alpha$ to nieskończony zbiór liczb całkowitych z powtórzeniami:
$$Spec(\alpha) = \{\lfloor\alpha\rfloor, \lfloor2\alpha\rfloor, \lfloor3\alpha\rfloor, ...\}.$$

Ilość elementów $Spec(\alpha)$ nie większych niż $n$ wynosi:
\begin{align*}
    N(\alpha, n)=\sum\limits_{k>0}(\lfloor k\alpha\leq n\rfloor)=\sum\limits_{k>0}(\lfloor k\alpha<n+1\rfloor)=\sum\limits_{k>0}(k\alpha<n+1)=\lceil {(n+1)\over\alpha}-1\rceil
\end{align*}

\subsection{Operacja mod}

Wzór na dzielenie liczby $n$ przez $m$:
$$n=m\Big\lfloor{n\over m}\Big\rfloor + n\mod m,$$
czyli
$$n\mod m=n-m\Big\lfloor{n\over m}\Big\rfloor,\quad m\neq 0$$
Jest to definicja która działa też dla liczb ujemnych, na przykłady
\begin{align*}
    5\mod 3&=5-3\Big\lfloor {5\over 3}\Big\rfloor = 2\\
    5\mod -3&=5-(-3)\Big\lfloor {5\over -3}\Big\rfloor = -1\\
    -5\mod-3&=-5-(-3)\Big\lfloor{-5\over-3}\Big\rfloor = -2
\end{align*}

Część ułamkową można zdefiniować jako operację modulo:
$$\{x\}=x\mod1.$$

Działania modulo:
$$c(x\mod y)=cx\mod cy$$

Jeśli rozmieściemy $n$ przedmiotów do $m$ grup tak, żeby różnica ich ilości między grupami była nie większa niż 1, a większe grupy były bliżej początku, to bez problemu możemy sprawdzić ilość przedmiotów w $k$-tej grupie:
$$\Big\lceil{n-k+1\over m}\Big\rceil.$$
A więc skoro mamy $n$ przedmiotów w $m$ grupach, to całość dostaniemy sumując wszystkie grupy:
$$n=\Big\lceil{n\over m}\Big\rceil+...+\Big\lceil{n-m+1\over m}\Big\rceil.$$
Po prostu powoli odejmujemy coraz więcej od reszty z tego dzielenia.\bigskip

Podobna zależność działa też dla liczb rzeczywistych:
$$\lfloor mx\rfloor = \lfloor x\rfloor+\Big\lfloor x+\frac1m\Big\rfloor...+\Big\lfloor x+{m-1\over m}\Big\rfloor$$

{\color{cyan}WYPADAŁOBY TUTAJ WRÓCIĆ, ALE CHWILOWO MAM DOŚĆ OPERACJI NA PODŁOGACH I INNYCH SUFITACH}