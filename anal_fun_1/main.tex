\documentclass{article}

\usepackage{../uni-notes}

\title {Analiza funkcjonalna}
\author{by a fish}
\date {21.03.2137}

\begin{document}

\maketitle

\newpage

\section*{SYLABUS:}
1. Twierdzenie Hahna-Banacha\\
\indent - Zbiory wypukłe w przestrzeniach wektorowych\\
\indent - Funkcjonał Minkowskiego\\
\indent - Półnorna i norma\\
\indent - Twierdzenie Hahna-Banacha\smallskip

2. Ogólna teoria przestrzeni unormowanych i przestrzeni Banacha\\
\indent - Przestrzenie unormowane\\
\indent - Pojęcie przestrzeni Banacha\\
\indent - Równoważność norm\\
\indent - Ośrodkowe przestrzenie Banacha\\
\indent - Przestrzenie produktowe iprzestrzenie ilorazowe\\
\indent - Ograniczone operatory liniowe\\
\indent - Norma operatora\\
\indent - Operatorry odwracalne\\
\indent - Izomorfizm\smallskip

3. Przestrzenie Hilberta\\
\indent - Pojęcie przestrzeni Hilberta\\
\indent - Twierdzenie o najlpeszej aproksymacji\\
\indent - Rzut ortogonalny i dopełnienie ortogonalne\\
\indent - Układy ortogonalne i nierówność Bessela\\
\indent - Baza ortonornalna i równość Parsevala\\
\indent - Proces ortogonalizacji Grama-Schmidta\\
\indent - Ośrodkowe przestrzenei Hilberta\\
\indent - Funkcjonały liniowe\\
\indent - Przestrzeń spzężona\smallskip

4. Twierdzenie Baire'a\\
\indent - Twierdzenie Baire'a\\
\indent - Twierdenie Banacha-Steinhausa\\
\indent - Twierdzenie Banacha o odwzorowaniu otwartym, odwzorowaniu odwrotnym i wykresie domkniętym\smallskip

5. Przestrzenie liniowe topologiczne\\
\indent - Topologie liniowe\\
\indent - Przestrzeń dualna\\
\indent - Operator sprzężony\\
\indent - Słabe topologie\\
\indent - Topologia *-słaba\\
\indent - Twierdzenie Banacha-Alaoglu\\
\indent - Refleksywność\smallskip

6. Twierdzenie Stone'a-Weierstrassa

\newpage

\tableofcontents

\newpage

\end{document}